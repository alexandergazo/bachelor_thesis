%%% The main file. It contains definitions of basic parameters and includes all other parts.

%% Settings for single-side (simplex) printing
% Margins: left 40mm, right 25mm, top and bottom 25mm
% (but beware, LaTeX adds 1in implicitly)
\documentclass[12pt,a4paper]{report}
	\usepackage[utf8]{inputenc}
	\usepackage[slovak, english]{babel}
	\usepackage{amsmath, amsthm, amssymb}
	\usepackage{graphicx, tabularx}
	\usepackage{mathtools}
	\usepackage{nomencl}
	\usepackage[font=small]{caption}
	\usepackage{epigraph}
	
	\setlength\textwidth{145mm}
	\setlength\textheight{247mm}
	\setlength\oddsidemargin{15mm}
	\setlength\evensidemargin{15mm}
	\setlength\topmargin{0mm}
	\setlength\headsep{0mm}
	\setlength\headheight{0mm}
	% \openright makes the following text appear on a right-hand page
	\let\openright=\clearpage

	%% Settings for two-sided (duplex) printing
	% \documentclass[12pt,a4paper,twoside,openright]{report}
	% \setlength\textwidth{145mm}
	% \setlength\textheight{247mm}
	% \setlength\oddsidemargin{14.2mm}
	% \setlength\evensidemargin{0mm}
	% \setlength\topmargin{0mm}
	% \setlength\headsep{0mm}
	% \setlength\headheight{0mm}
	% \let\openright=\cleardoublepage

	%% Generate PDF/A-2u
	\usepackage[a-2u]{pdfx}

	%% Character encoding: usually latin2, cp1250 or utf8:
	\usepackage[utf8]{inputenc}

	%% Prefer Latin Modern fonts
	\usepackage{lmodern}

	%% Further useful packages (included in most LaTeX distributions)
	\usepackage{amsmath}        % extensions for typesetting of math
	\usepackage{amsfonts}       % math fonts
	\usepackage{amsthm}         % theorems, definitions, etc.
	%\usepackage{bbding}         % various symbols (squares, asterisks, scissors, ...)
	\usepackage{bm}             % boldface symbols (\bm)
	\usepackage{graphicx}       % embedding of pictures
	\usepackage{fancyvrb}       % improved verbatim environment
	\usepackage[round]{natbib}         % citation style AUTHOR (YEAR), or AUTHOR [NUMBER]
	\usepackage[nottoc]{tocbibind} 
					% makes sure that bibliography and the lists
				    % of figures/tables are included in the table
				    % of contents
	\usepackage{dcolumn}        % improved alignment of table columns
	\usepackage{booktabs}       % improved horizontal lines in tables
	\usepackage{paralist}       % improved enumerate and itemize
	%\usepackage[usenames]{xcolor}  % typesetting in color

	%%% Basic information on the thesis

	% Thesis title in English (exactly as in the formal assignment)
	\def\ThesisTitle{Pole Shifting Theorem in Control Theory}
	\def\ThesisTitleSk{Věta o přiřazení pólů v teorii řízení}

	% Author of the thesis
	\def\ThesisAuthor{Alexander Gažo}
	
	% Year when the thesis is submitted
	\def\YearSubmitted{2019}
	
	% Name of the department or institute, where the work was officially assigned
	% (according to the Organizational Structure of MFF UK in English,
	% or a full name of a department outside MFF)
	\def\Department{Department of Algebra}
	\def\DepartmentSk{Katedra algebry}
	
	% Is it a department (katedra), or an institute (ústav)?
	\def\DeptType{Department}
	\def\DeptTypeSk{Katedra}
	
	% Thesis supervisor: name, surname and titles
	\def\Supervisor{doc. RNDr. Jiří Tůma, DrSc.}
	
	% Supervisor's department (again according to Organizational structure of MFF)
	\def\SupervisorsDepartment{Department of Algebra}
	\def\SupervisorsDepartmentSk{Katedra algebry}
	
	% Study programme and specialization
	\def\StudyProgramme{Mathematics}
	\def\StudyBranch{Mathematical Structures}

	% An optional dedication: you can thank whomever you wish (your supervisor,
	% consultant, a person who lent the software, etc.)
	\def\Dedication{%
		I would like to thank doc. RNDr. Jiří Tůma, DrSc. for always pleasant consultations, valuable advice and patience. I would also like to thank Peter Guba for his time in assisting me with the English side of the thesis.
	}

	% Abstract (recommended length around 80-200 words; this is not a copy of your thesis assignment!)
	\def\Abstract{%
		In this thesis, I describe the notions needed for understanding and proving the pole-shifting theorem, as well as the theorem itself. I do this by frist defining first order dynamical linear systems with constant coefficients with control and defining controllability of such a system. 
		Then I show that the definition of controllability motivated by discrete-time systems also holds for continuous-time systems.
		Using these notions, the pole-shifting theorem can be formulated and proved.
	}
	\def\AbstractSk{%

	}

	% 3 to 5 keywords (recommended), each enclosed in curly braces
	\def\Keywords{%
		{discrete linear dynamical system with constant coefficients},
		{continuous linear dynamical system with constant coefficients},
		{eigenvalue assignment},
		{control},
		{controllability},
		{linear feedback},
		{stability},
		{basic control theory}
	}
	\def\KeywordsSk{%
		{diskrétny lineárny dynamický systém s konštantnými koeficientmi},
		{spojitý lineárny dynamický systém s konštantnými koeficientmi},
		{priradenie vlastných čísiel},
		{riadenie},
		{kontrolovateľnosť},
		{stabilita},
		{základy teórie riadenia}
	}

	\hypersetup{unicode}
	\hypersetup{breaklinks=true}
	%\usepackage[pdftex,unicode]{hyperref}   % Must follow all other packages
	%\hypersetup{breaklinks=true}
	%\hypersetup{pdftitle={\ThesisTitle}}
	%\hypersetup{pdfauthor={\ThesisAuthor}}
	%\hypersetup{pdfkeywords=\Keywords}
	%\hypersetup{urlcolor=blue}

\newcommand{\R}{\mathbb{R}}   
\newcommand{\Z}{\mathbb{Z}}   
\newcommand{\N}{\mathbb{N}}   
\renewcommand{\C}{\mathbb{C}}
\newcommand{\K}{\mathbb{K}}
\newcommand{\powerset}[1]{\mathcal{P} ( #1 )}   
\newcommand{\termdef}[1]{\textnormal{\textbf{#1}}}
\newcommand{\nullvector}{\textit{o}}

\DeclarePairedDelimiter\abs{\lvert}{\rvert}%
\DeclarePairedDelimiter\norm{\lVert}{\rVert}%

\makeatletter
\newcommand*{\Relbarfill@}{\arrowfill@\Relbar\Relbar\Relbar}
\newcommand*{\xeq}[2][]{\ext@arrow 0055\Relbarfill@{#1}{#2}}
\makeatother

\newsavebox{\overlongequation}
\newenvironment{longeq}
 {\begin{displaymath}\begin{lrbox}{\overlongequation}$\displaystyle}
 {$\end{lrbox}\makebox[0pt]{\usebox{\overlongequation}}\end{displaymath}}

% Swap the definition of \abs* and \norm*, so that \abs
% and \norm resizes the size of the brackets, and the 
% starred version does not.
\makeatletter
\let\oldabs\abs
\def\abs{\@ifstar{\oldabs}{\oldabs*}}
%
\let\oldnorm\norm
\def\norm{\@ifstar{\oldnorm}{\oldnorm*}}
\makeatother

\makeatletter
\renewcommand*\env@matrix[1][*\c@MaxMatrixCols c]{%
  \hskip -\arraycolsep
  \let\@ifnextchar\new@ifnextchar
  \array{#1}}
\makeatother

\newtheorem{theorem}{Theorem}[]
\newtheorem{lemma}{Lemma}[]
\newtheorem{cor}{Corollary}[]
\newtheorem{claim}{Claim}[]
\newtheorem*{definition}{Definition}
\newtheorem*{remark}{Remark}
\newtheorem{example}{Example}

\newcommand{\bigO}{\mathcal{O}}

\begin{document}
\include{title}

%%% A page with automatically generated table of contents of the bachelor thesis

\tableofcontents

%%% Each chapter is kept in a separate file
\chapter*{Introduction}
\addcontentsline{toc}{chapter}{Introduction}

The pole shifting theorem is one of the basic results of the theory of linear dynamical systems with linear feedback. It claims that in case of controllable systems one can achieve an arbitrary asymptotic behavior by a suitably chosen feedback. To understand this crucial theorem, we must first describe few basic concepts.
\chapter{Introduction}
\label{chap:intr}

\section{Basics}
\label{sec:basics}

Pole shifting theorem deals with linear differential systems and claims, that it is possible to achieve arbitrary asymptotic behavior. To understand this basic theorem of control theory, we must first describe few basic concepts.

\subsection{Systems of First Order Differential Equations}

\begin{definition}
	Let us have a system of linear differential equations of order of one with constant coefficients. Then a \termdef{matrix differential equation} for this system is in the form $$\dot{x}(t)=Ax(t)$$ 
\end{definition}

We will use this representation as it a very compact way of describing the system.

To express solution of this system in similarly compact matter we will establish a notion of matrix exponential.

\begin{definition}
	Let $X$ be real or complex square matrix. The exponential of $X$, denoted by $e^X$, is the square matrix of same dimensions given by the power series $$e^{X}=\sum _{k=0}^{\infty}\frac{1}{k!}X^{k}$$
	where $X^0$ is defined to be the identity matrix $I$ with the same dimensions as $X$.
\end{definition}

\begin{remark}
\label{rem:expprop}
	Let $A$, $B$ and $X$ be real or complex square matrices of same dimensions. Then 
	\begin{enumerate}
		\item $e^X$ converges for any matrix $X$.
		\item $\frac{d}{dt}e^{Xt}=Xe^{Xt}$, for $t \in \R$
		\item $AB = BA \Rightarrow e^{A+B} = e^{A}e^B$
		\item $AB = BA \Rightarrow e^{A}B = Be^{A}$
		\item If $R$ is invertible then $e^{R^{-1}XR}=R^{-1}e^AR$
	\end{enumerate}
\end{remark}

\begin{proof}
	\begin{enumerate}
		\item We want to show that the partial sums $S_N=\sum^N_{k=0}\frac{1}{k!}X^k$ converge to $e^X$. This can be shown by choosing any matrix norm satisfying $||AB||<||A||\cdot||B||$ and writing $$||e^X-S_N||=||\sum^\infty_{k=N+1}\frac{1}{k!}X^k||=\sum^\infty_{k=N+1}\frac{1}{k!}||X||^k$$ where right side approaches zero since $\sum^\infty_{k=0}\frac{||A||^k}{k!}$ converges.

		\item Thanks to the convergence we have $$\frac{d}{dt}e^{Xt}=\frac{d}{dt}\sum^\infty_{k=0}\frac{t^k}{k!}X^{k}=\sum^\infty_{k=0}\frac{t^k}{k!}X^{k+1}=X\sum^\infty_{k=0}\frac{t^k}{k!}X^{k}=Xe^{Xt}$$

		\item Let us write $$e^{A+B}=\sum^\infty_{k=0}\frac{1}{k!}(A+B)^{k}=\sum^\infty_{k=0}\sum^k_{l=0}\binom{k}{l}\frac{1}{k!}A^{l}B^{k-l}=\sum^\infty_{k=0}\frac{1}{k!}A^{k}\cdot\sum^\infty_{k=0}\frac{1}{k!}B^{k}=e^{A}e^B$$ In the second equation we are using the assumption $AB=BA$ and in the third equation the Cauchy product formula is used.

		\item From definition follows $$e^{A}B=\sum^\infty_{k=0}\frac{1}{k!}A^{k}B\stackrel{AB=BA}{=}\sum^\infty_{k=0}\frac{1}{k!}BA^{k}=B\sum^\infty_{k=0}\frac{1}{k!}A^{k}=Be^{A}$$ 
		
		\item We have $$e^{R^{-1}XR}=\sum^\infty_{k=0}\frac{1}{k!}(R^{-1}XR)^{k}=\sum^\infty_{k=0}\frac{1}{k!}R^{-1}X^{k}R=R^{-1}(\sum^\infty_{k=0}\frac{1}{k!}X^{k})R=R^{-1}e^{X}R$$ 
	\end{enumerate}
\end{proof}

Now, using properties from Remark \ref{rem:expprop} we can see that $\dot{x}(t)=Ax(t)$ is actually solved by $x(0)e^At$. Let us now discuss under what circumstances does this system converge to $\nullvector$. 

Let $A$ be real or complex matrix. Then we have $$J=R^{-1}AR$$ where $J$ is a matrix in the Jordan normal form. Therefore we can write $$\dot{x}(t)=RR^{-1}ARR^{-1}x(t)=RJR^{-1}x(t)$$ It follows that $$R^{-1}\dot{x}(t)=\dot{(R^{-1}x)}=J R^{-1}x(t)$$ By substituting $y(t)=R^{-1}x(t)$, which is equivalent with changing the basis of our system, we get $$\dot{y}(t)=Jy(t)$$ and therefore the solution is $$y(t)=e^{Jt}y(0)$$ 

We know that every Jordan block $J_{\lambda,n}$ in the matrix $J$ can be decomposed as $J_{\lambda,n}=\lambda I_n+N_n$, $n \in\N$ where $N_n$ is $n \times n$ nilpotent matrix satisfying $n_{i,j}=\delta_{i,j-1}$. For example for $n=4$ we have
\begin{equation*}
	N_4=
	\begin{pmatrix}
		0 & 1 & 0 & 0 \\
		0 & 0 & 1 & 0 \\
		0 & 0 & 0 & 1 \\
		0 & 0 & 0 & 0 
	\end{pmatrix}
\end{equation*}
It is also true that $(N_n)^n=\textbf{O}$, since every right multiplication by matrix $N$ shifts the multiplied matrix's columns to the right by one column. 

By using Remark \ref{rem:expprop}, we now for each Jordan block $J_{\lambda,n}$ have $$e^{J_{\lambda,n}t}=e^{(\lambda I + N)t}=e^{\lambda It}e^{Nt}=e^{\lambda t}e^{Nt}$$ Let $\lambda = a+bi$ where $a$,$b \in \R$. Then we have $$e^{J_nt}=e^{at}e^{bit}e^{Nt}$$ We know that $|e^{bit}|=1$ and that $$e^{Nt}=\sum^{n-1}_{k=0}\frac{t^k}{k!}N^k$$ so the highest power of $t$ in the matrix is $n-1$. Therefore, we can see that the whole expression approaches 0 in infinity if $$\lim_{t\to\infty}e^{at}t^{n-1}=0$$ This holds for any $n\in\N$ as long as $a<0$. 

Now, since $y(0)$ is a constant vector, we see that $y(t)=e^{Jt}y(0)$ converges to 0 if all the eigenvalues of matrix $A$ are negative in their real parts.

\subsection{Omg}

The state of linear system can be represented by system of $n$ differential equations $$\dot{x}(t)=Ax(t)+Bu(t),$$ where $x(t)$ is the $n$-dimensional state vector $(\varphi(t),\dot{\varphi}(t),\ldots,\varphi^{(n-1)}(t))^T$, $u(t)$ is the $m$-dimensional \textit{input} or \textit{control vector}, $A \in \C ^{n\times n}$ is matrix of coefficients and $B\in \C ^{m\times n}$ is \textit{input} or \textit{control matrix}. The whole system is called \textit{n-dimensional system}. The control vector $u(t)$ is acquired from $x(t)$ by multiplying a control matrix $F\in \C ^{m\times n}$ by $x(t)$. All the possible states of $x(t)$ create \textbf{state space} which is usually equal to $\C^n$. 

We can imagine this system as follows. The first part of the equation $\dot{x}(t)=Ax(t)$ can be thought of as the model of machine or event that we want to control and $Bu(t)$ as our control mechanism. The $B$ matrix is our ``control board'' and $u(t)$ is us deciding, which levers and buttons we want to push. Of course, if we want this system to be self-regulating, we cannot input our own values into $u(t)$ and therefore it has to be calculated from the current state of our system. Thus $u(t)=Fx(t)$. The whole system can then be rewritten as $$\dot{x}(t)=Ax(t)+BFx(t)=(A+BF)x(t).$$ If $A+BF$ is a diagonalizable matrix, then we have $$\Lambda=R^{-1}(A+BF)R,$$ where $R \in \C ^{n \times n}$ is an invertible matrix and $\Lambda \in \C ^{n \times n}$ is a diagonal matrix. We can write that $$\dot{x}(t)=RR^{-1}(A+BF)RR^{-1}x(t)=R\Lambda R^{-1}x(t),$$ it follows, that $$R^{-1}\dot{x}(t)=\dot{(R^{-1}x)}=\Lambda R^{-1}x(t).$$ By substituting $y(t)=R^{-1}x(t)$ we get $$\dot{y}(t)=\Lambda y(t).$$ This equation represents system of simple linear differential equations. If we denote elements on diagonal of $\Lambda$ by $\lambda_1,\lambda_2,\ldots,\lambda_n$ the resulting equations are  
\begin{align*}
  \dot{y}_1(t)&=\lambda_1y_1(t) \\
  \dot{y}_2(t)&=\lambda_2y_2(t) \\
  &\vdotswithin{=} \\
  \dot{y}_n(t)&=\lambda_ny_n(t) 
\end{align*}
Solution to each of these equations is in the form 
$$y_k(t)=y_k(0)e^{\lambda_kt}, k\in\{1,2,\ldots,n\}.$$
Let $\lambda_k=a_k+b_ki$ where $a_k,b_k\in \R$, then 
$$y_k(0)e^{\lambda_kt}=y_k(0)e^{a_kt}e^{b_kit}.$$ 
We know, that $|e^{b_kit}|=1$ and that $y_k(0)$ is a constant, so the crucial part is $e^{a_kt}$. This converges to 0 if and only if $a_k$ is negative. Therefore we can stabilize our ``machine'' if we find such matrix $F \in \C^{n \times n}$ that $A+BF$ is diagonalizable with distinct eigenvalues (then we are sure that $A+BF$ is diagonalizable) with negative real part. This can be expressed through characteristic polynomial of matrix $A+BF$. We will denote characteristic polynomial of a matrix $A$ by $\chi_A$. Through our observations we got to a conclusion that we need to satisfy $$\chi_{A+BF}=(x-\lambda_1)(x-\lambda_2)\cdots(x-\lambda_n),$$ where $\lambda_1,\lambda_2,\ldots,\lambda_n \in \C$ are different from one another and their real parts are negative. This leads to an important definition.

\begin{definition}
    Let $\K$ be a field and let $A \in \K^{n \times n}$, $B \in \K^{n \times m}$, $n,m \in \N$. We say that polynomial $\chi$ is \termdef{assignable} for the pair $(A,B)$ if there exists such matrix $F\in\K^{m \times n}$ that $$\chi_{A+BF}=\chi$$
\end{definition}

The pole shifting theorem states, that if $A$ and $B$ are ``sensible'' in a sense that we will discuss in the next section, then arbitrary polynomial $\chi$ of dimension that depends on how ``sensible'' $A$ and $B$ are, can be assigned to the pair $(A,B)$. It also claims that it is immaterial over what field $A$ and $B$ are.

\section{Controllable pairs}

In this section we will establish the notion of controllability. We will first explain this concept for \textit{discrete-time systems} and then we will show that the requirement for controllability for \textit{discrete-time systems} also holds for \textit{continuos-time systems}.

\subsection{Discrete-time systems}

States that we can reach in set number of iterations in a \textit{discrete-time systems} can be derived as follows. From state $x_k$ and control vector $u_k$ is the next state $x_{k+1}$ computed by equation $$x_{k+1}=Ax_k+Bu_k$$ where $\K$ is a field, $A\in\K^{n\times n}$ and $B\in\K^{n\times m}$. The starting condition is $x_0=\textbf{o}$ and we can choose arbitrary $u_k$. Then, for $k=0$ we have $$x_1=Ax_0+Bu_0=Bu_0 \in \text{Im}B.$$ For $k=1$ we get $$x_2=Ax_1+Bu_1=ABu_0+Bu_1\in\text{Im}(AB|B).$$ It is clear, that $$x_k\in\text{Im}(A^{k-1}B|\ldots|AB|B).$$ We can observe that $\text{Im}(B|AB|\ldots|A^kB) \subseteq \text{Im}(B|AB|\ldots|A^{k+1}B)$. Then, from Cayley-Hamilton theorem we know that $$\text{Im}(B|AB|\ldots|A^{n-1}B)=\text{Im}(B|AB|\ldots|A^{n-1}B|A^nB).$$ Therefore all the states we could ever reach are already in space $$\text{Im}(B|AB|\ldots|A^{n-1}B).$$

\begin{definition}
	Let $\K$ be a field and let $A \in \K^{n \times n}$, $B \in \K^{n \times m}$, $n,m \in \N$. We define \termdef{reachable space} $\mathcal{R}(A,B)$ as $\text{Im}(B|AB|\ldots|A^{n-1}B)$.
\end{definition}

\begin{definition}
	Let $V$ be a vector space, $W$ be its subspace and $f$ be a mapping from $V$ to $V$. We call $W$ an invariant subspace of $f$ if $f(W)\subseteq W$. 

	We also say that $W$ is $f$-invariant.
\end{definition}

\begin{lemma}
	\label{lem:invmatrix}
	Let $W$ be an invariant subspace of linear mapping $f\colon V \rightarrow V$. Then there exists a basis $C$ of $V$ such that 
	\begin{equation*}
		[f]^C_C=
		\begin{pmatrix}
			F_1 & F_2 \\
			0   & F_3 
		\end{pmatrix}
	\end{equation*}
	where $F_1$ is $r\times r$, $r=\text{dim}W$.
\end{lemma}

\begin{proof}
	We have $$[f]^C_C=[\text{id}]^K_C [f]^K_K [\text{id}]^C_K=([\text{id}]^C_K)^{-1} [f]^K_K [\text{id}]^C_K$$ where $[id]^C_K$ is a transition matrix from basis $C$ to canonical basis $K$. Let $(w_1,\ldots,w_r)$ be an arbitrary basis of subspace $W$. We complete this sequence into basis of $V$ with vectors $v_1,\ldots,v_{n-r}$ where $n=\text{dim}V$. We now put $$C=(w_1,\ldots,w_r,v_1,\ldots,v_{n-r})$$ Since $W$ is $f$-invariant we know that $[f]^K_K [\text{id}]^C_K = (u_1,\ldots,u_r,z_1,\ldots,z_{n-r})$, where $u_i\in W$ and $z_i\in V$. Now, by left multiplying this result by $[\text{id}]^K_C$ we get the matrix $([u_1]_C,\ldots,[u_r]_C,[z_1]_C,\ldots,[z_{n-r}]_C)$. Thanks to our choice of basis $C$ can any vector $u_i$ be expressed as a linear combination of vectors $(w_1,\ldots,w_r)$ and therefore we now have the desired form.
\end{proof}

From discussion above it is clear that for arbitrary $v\in\mathcal{R}(A,B)$ we have $Av\in\mathcal{R}(A,B)$. This property is called $A$-\textit{invariance}.

The maximum dimension of $\mathcal{R}(A,B)$ is, of course, $n$. This leads us to important property of pair $(A,B)$, where we want to able to get the ``machine'' into any state in state space by controlling it with our control matrix $B$. Therefore we desire that $\mathcal{R}(A,B)=\K^n$. The equivalent condition is $\text{dim}\mathcal{R}(A,B)=n$.

\begin{definition}
	Let $\K$ be a field and let $A \in \K^{n \times n}$, $B \in \K^{n \times m}$, $n,m \in \N$. The pair $(A,B)$ is \termdef{controllable} if $\textnormal{dim}\mathcal{R}(A,B)=n$.
\end{definition}

\subsection{Continuous-time systems}

We will now show that the condition for \textit{discrete-time systems} is also characterizing for \textit{continuous-time systems}. For this we have to express solution of such system using matrices $A^iB$ for $i\in \N_0$. 

\begin{definition}
	Let $X$ be real or complex square matrix. The exponential of $X$, denoted by $e^X$, is the square matrix of same dimensions given by the power series $$e^{X}=\sum _{k=0}^{\infty}\frac{1}{k!}X^{k}$$
	where $X^0$ is defined to be the identity matrix $I$ with the same dimensions as $X$.
\end{definition}

We will need few properties of this notion which we will form in the next claim. 

\begin{remark}
	Let $A$, $B$ and $X$ be real or complex square matrices of same dimensions. Then 
	\begin{enumerate}
		\item $e^X$ converges for any matrix $X$.
		\item $\frac{d}{dt}e^{Xt}=Xe^{Xt}$, for $t \in \R$
		\item $AB = BA \Rightarrow e^{At}B = Be^{At}$, for $t \in \R$
	\end{enumerate}
\end{remark}

\begin{proof}
	\begin{enumerate}
		\item We want to show that the partial sums $S_N=\sum^N_{k=0}\frac{1}{k!}X^k$ converge to $e^X$. This can be shown by choosing any matrix norm satisfying $||AB||<||A||\cdot||B||$ and writing $$||e^X-S_N||=||\sum^\infty_{k=N+1}\frac{1}{k!}X^k||=\sum^\infty_{k=N+1}\frac{1}{k!}||X||^k$$ where right side approaches zero since $\sum^\infty_{k=0}\frac{||A||^k}{k!}$ converges.

		\item Thanks to the convergence we have $$\frac{d}{dt}e^{Xt}=\frac{d}{dt}\sum^\infty_{k=0}\frac{t^k}{k!}X^{k}=\sum^\infty_{k=0}\frac{t^k}{k!}X^{k+1}=X\sum^\infty_{k=0}\frac{t^k}{k!}X^{k}=Xe^{Xt}$$

		\item From definition follows $$e^{At}B=\sum^\infty_{k=0}\frac{t^k}{k!}A^{k}B\stackrel{AB=BA}{=}\sum^\infty_{k=0}\frac{t^k}{k!}BA^{k}=B\sum^\infty_{k=0}\frac{t^k}{k!}A^{k}=Be^{At}$$ 
	\end{enumerate}
\end{proof}

Now we utilize matrix exponential in solving system of inhomogeneous linear system $\dot{x}(t)=Ax(t)+Bu(t)$. By left multiplying it by $e^{-tA}$ we get
\begin{align*}
	e^{-tA}\dot{x}(t)-e^{-tA}Ax(t) &=e^{-tA}Bu(t) \\
	\frac{d}{dt} (e^{-tA}x(t)) &=e^{-tA}Bu(t) 
\end{align*}
Note we used $-AA=A(-A)\Rightarrow e^{-tA}A=Ae^{-tA}$. After integrating both sides with respect to $t$ on interval $(t_0,t_1)$ we have 
\begin{align*}
	[e^{-tA}x(t)]^{t_1}_{t_0}&=\int^{t_1}_{t_0}e^{-tA}Bu(t)dt \\
	e^{-t_1A}x(t_1)-e^{-t_0A}x(t_0)&=\int^{t_1}_{t_0}e^{-tA}Bu(t)dt \\
	x(t_1)&=e^{(t_1-t_0)A}x(t_0)+\int^{t_1}_{t_0}e^{(t_1-t)A}Bu(t)dt
\end{align*}

Now it is clear that in system where $x(0)=\nullvector$ can every state in time $t\in \R^+$ be expressed as $$x(t)=\int^t_0 e^{(t-s)A}Bu(s)ds$$

\begin{theorem}
	The $n$-dimensional continuos-time linear system is controllable if and only if $\text{dim}\mathcal{R}(A,B)=n$.
\end{theorem}

\begin{proof}
	From discussion above we have $$x(t)=\int^t_0e^{(t-s)A}Bu(s)ds=\int^t_0\sum^\infty_{k=0}\frac{(t-s)^k}{k!}A^kBu(s)$$ we can see that $$x(t) \in \text{Im}(B|AB|\ldots|A^kB|\ldots)\subseteq \text{Im}(B|AB|\ldots|A^{n-1}B)=\mathcal{R}(A,B)$$ 
	
	If the system is controllable then $x(t)$ can be equal to any of the vectors of an arbitrary basis of $\K^n$. Therefore we know that $n$ linearly independent vectors belong into $\mathcal{R}(A,B)$ and naturally it follows $\text{dim}\mathcal{R}(A,B)=n$.

	Conversely, if dimension of reachable space is less than $n$, then any basis vector of complement space to $\mathcal{R}(A,B)$ cannot be equal to $x(t)$. Therefore the system is not controllable.
\end{proof}

\subsection{Decomposition theorem}

If $(A,B)$ are not controllable then there exists subspace of our state space that is not affected by our input. This can be shown using following theorem.

\begin{theorem}
	\label{theorem:decomp}
	Assume that $(A,B)$ is not controllable. Let $\text{dim}\mathcal{R}(A,B)=r<n$. Then there exists invertible $n\times n$ matrix $T$ over $\K$ such that the matrices $\widetilde{A}:=T^{-1}AT$ and $\widetilde{B}:=T^{-1}B$ have the block structure 
	\begin{equation*}
		\widetilde{A}=
		\begin{pmatrix}
			A_1 & A_2 \\
			0   & A_3 
		\end{pmatrix}
		\qquad
		\widetilde{B}=
		\begin{pmatrix}
			B_1  \\
			0
		\end{pmatrix}
	\end{equation*}
	where $A_1$ is $r \times r$ and $B_1$ is $r \times m$.
\end{theorem}

\begin{proof}
	Let $\mathcal{S}$ be any subspace that $$\mathcal{R}(A,B)\oplus\mathcal{S}=\K^n.$$ Let $\{v_1,\ldots,v_r\}$ be the basis of $\mathcal{R}(A,B)$ and $\{w_1,\ldots,w_{n-r}\}$ be the basis of $\mathcal{S}$, then we put $K=(v_1,\ldots,v_r,w_1,\ldots,w_{n-r})$ as the basis of $\K^n$ and we put $$T:=(v_1|\ldots|v_r|w_1|\ldots|w_{n-r})=[\text{id}]^K_C$$ where $C$ is the canonical basis and $[\text{id}]^K_C$ is the transition matrix from basis $K$ to basis $C$. We have $\text{Im}T=\K^n$ therefore $T$ is an invertible matrix. It holds $$\widetilde{B}=T^{-1}B=([\text{id}]^K_C)^{-1}B=[\text{id}]^C_KB$$ We know that $\text{Im}B\subseteq\mathcal{R}(A,B)$ therefore every column of matrix $B$ can be uniquely expressed as linear combination of vectors in basis $K$. From our choice of $T$ we can clearly see that $\widetilde{B}$ will be of the desired form.
	
	As for $\widetilde{A}$ we have $$\widetilde{A}=T^{-1}AT=[\text{id}]^C_KA[\text{id}]^K_C$$ From the fact that $\mathcal{R}(A,B)$ is $A$-invariant it follows that $$AT=(u_1|\ldots|u_r|z_1|\ldots|z_{n-r})$$ where $u_i \in \mathcal{R}(A,B)$ and $z_i \in \K^i$. Therefore, when we express these vectors in the basis $K$ (by left multiplying $AT$ by $T^{-1}=[\text{id}]^C_K$) we get the required structure of matrix $\widetilde{A}$.
\end{proof}

We achieved the new form of matrices $A$ and $B$ by changing the basis of our state space. We define relation between $(A,B)$ and $(\widetilde{A},\widetilde{B}).$

\begin{definition}
	Let $(A,B)$ and $(\widetilde{A},\widetilde{B})$ be pairs as above. Then $(A,B)$ \termdef{is similar to} $(\widetilde{A},\widetilde{B})$, denoted $$(A,B) \sim (\widetilde{A},\widetilde{B})$$ if there exists invertible matrix $T$ for which it holds that $$\widetilde{A}=T^{-1}AT\quad and\quad\widetilde{B}=T^{-1}B$$
\end{definition}

We can interpret the decomposition as follows. Consider our system $\dot{x}(t)=Ax(t)+Bu(t)$. By changing the basis by putting $x(t)=Ty(t)$ we get $$T\dot{y}(t)=ATy(t)+Bu(t)$$ which we can write as $$\dot{y}(t)=T^{-1}ATy(t)+T^{-1}Bu(t)=\widetilde{A}y(t)+\widetilde{B}u(T)$$ Which gives us 
\begin{alignat*}{2}
	\dot{y}_1(t)&=A_1y_1(t)+&A_2y_2(t)&+B_1u_1(t) \\
	\dot{y}_2(t)&=&A_3y_2(t)&
\end{alignat*}
where $y_1(t)$ is the first $r$ elements of $y(t)$, $y_2(t)$ is the other $n-r$ elements of $y(t)$ and $u_1(t)$ is the first $r$ elements of $u(t)$. It is clear that we cannot control $\dot{y}_2$ by any means. 

It is also true that $(A_1,B_1)$ is a controllable pair which we will state in a lemma.

\begin{lemma}
	The pair $(A_1,B_1)$ is controllable.
\end{lemma}

\begin{proof}
	We know that $\text{dim}\mathcal{R}(A,B)=r$. We desire $\text{dim}\mathcal{R}(A_1,B_1)=r$. We will show that $\mathcal{R}(\widetilde{A},\widetilde{B})=\mathcal{R}(A,B)$ and that each vector in $\mathcal{R}(\widetilde{A},\widetilde{B})$ has its last $n-r$ elements equal to 0 and that $\mathcal{R}(\widetilde{A},\widetilde{B})$ restricted on its first $r$ coordinates is equal to $\mathcal{R}(A_1,B_1)$. 
	\begin{align*}
		\mathcal{R}(\widetilde{A},\widetilde{B})&=\text{Im}(\widetilde{A}^{n-1}\widetilde{B}|\ldots|\widetilde{A}\widetilde{B}|\widetilde{B}) \\
		&=\text{Im}((T^{-1}AT)^{n-1}T^{-1}B|\ldots|T^{-1}ATT^{-1}B|T^{-1}B) \\
		&=\text{Im}(T^{-1}A^{n-1}B|\ldots|T^{-1}AB|T^{-1}B) \\
		&=T^{-1}\cdot\text{Im}(A^{n-1}B|\ldots|AB|B) \\
		&=T^{-1}\cdot\mathcal{R}(A,B)
	\end{align*}
	Since $T$ is an invertible matrix, which preserves linear independence, we have $$\text{dim}\mathcal{R}(\widetilde{A},\widetilde{B})=\text{dim}(T^{-1}\mathcal{R}(A,B))=\text{dim}(\mathcal{R}(A,B))=r$$

	Now let us focus on the structure of $\mathcal{R}(\widetilde{A},\widetilde{B})$: We know that last $n-r$ rows of $\widetilde{B}$ are $\nullvector$. Also because of structure of $\widetilde{A}$ we have for an arbitrary matrix $X\in\K^{r\times m}$ that 
	\begin{equation*}
		\widetilde{A}
		\begin{pmatrix}
			X \\
			0
		\end{pmatrix}
		=
		\begin{pmatrix}
			A_1 & A_2 \\
			  0 & A_3
		\end{pmatrix}
		\begin{pmatrix}
			X \\
			0
		\end{pmatrix}
		=
		\begin{pmatrix}
			A_1X \\
			0
		\end{pmatrix}
	\end{equation*}
	where again are the last $n-r$ rows equal to $\nullvector$. Therefore we see that for any positive integer $k$ we have 
	\begin{equation*}
		\widetilde{A}^k\widetilde{B}=
		\begin{pmatrix}
			A_1^{k}B_1 \\
			0
        \end{pmatrix}
        ,A_1^kB_1\in\K^{r\times r}
    \end{equation*}
    It follows
    \begin{equation*}
        \mathcal{R}(\widetilde{A},\widetilde{B})=
        \begin{pmatrix}[c|c|c|c]
            \begin{pmatrix}
                A_1^{n-1}B_1 \\
                0 
            \end{pmatrix}
            & \ldots &
            \begin{pmatrix}
                A_1B_1 \\
                0 
            \end{pmatrix}
            &
            \begin{pmatrix}
                B_1 \\
                0 
            \end{pmatrix}
        \end{pmatrix}
    \end{equation*}
	
	From Cayle-Hemilton theorem we therefore again have that the restriction to first $r$ coordinates (those which are not 0) of $\mathcal{R}(\widetilde{A},\widetilde{B})$ are equal to $\mathcal{R}(A_1,B_1)$. Finally, it follows that $$\text{dim}\mathcal{R}(A_1,B_1)=\text{dim}\mathcal{R}(\widetilde{A},\widetilde{B})=\text{dim}\mathcal{R}(A,B)=r$$
\end{proof}

Now we can see that the decomposition from Theorem \ref{theorem:decomp} decomposes the matrix $A$ into ``controllable'' and ``uncontrollable'' parts $A_1$ and $A_3$ respectively.

We also see that 
\begin{align*}
	\chi_{\widetilde{A}}&=\text{det}(sI-\widetilde{A})=\text{det}(sI-T^{-1}AT) \\
	&=\text{det}(sT^{-1}IT-T^{-1}AT)=\text{det}(T^{-1}(sI-A)T) \\
	&=(\text{det}T)^{-1}\text{det}(sI-A)\text{det}T=\text{det}(sI-A) \\
	&=\chi_A
\end{align*}
therefore it holds $$\chi_A=\chi_{A_1}\chi_{A_3}$$ 

\begin{definition}
	The characteristic polynomial of matrix $A$ splits into \termdef{controllable} and \termdef{uncontrollable parts} with respect to pair $(A,B)$ which we denote by $\chi_c$ and $\chi_u$ respectively. We define these polynomials as $$\chi_c=\chi_{A_1} \qquad \chi_u=\chi_{A_3}$$ In case $r=0$ we put $\chi_c=1$ and in case $r=n$ we put $\chi_u=1$.
\end{definition}
\chapter{The Pole Shifting Theorem}

The following chapter is based on the first section of the fifth chapter of \citet{Sontag1998}.

\begin{definition}
    The \termdef{controller form} associated to the pair $(A,b)$ is the pair 
    \begin{equation*}
        A^\flat=
        \begin{pmatrix}
            0 & 1 & 0 & \cdots & 0 \\
            0 & 0 & 1 & \cdots & 0 \\
            \vdots & \vdots & \vdots & \ddots & \vdots \\
            0 & 0 & 0 & \cdots & 1 \\
            \alpha_1 & \alpha_2 & \alpha_3 & \cdots & \alpha_n \\
        \end{pmatrix},
        \quad
        b^\flat=
        \begin{pmatrix}
            0 \\
            0 \\
            \vdots \\
            0 \\
            1
        \end{pmatrix}
    \end{equation*}
    where $s^n-\alpha_ns^{n-1}-\ldots-\alpha_2s-\alpha_1$ is the characteristic polynomial of $A$.
\end{definition}

\begin{lemma}
\label{lem:flatCharPol}
    The characteristic polynomial of $A^\flat$ is $s^n-\alpha_ns^{n-1}-\ldots-\alpha_2s-\alpha_1$.
\end{lemma}

\begin{proof}
    It can be shown using simple properties of the matrix determinant.
\end{proof}

\begin{lemma}
\label{lem:flatControllable}
    The pair $(A^\flat,b^\flat)$ is controllable.
\end{lemma}

\begin{proof}
    Because of the form of the vector $b^\flat$, the matrix $(A^\flat)^kb^\flat$ is equal to the last column of $(A^\flat)^k$, that is
    \begin{equation*}
        \begin{pmatrix}
            0 &
            0 &
            \cdots &
            0 &
            1 &
            \beta_{k-1} &
            \cdots &
            \beta_1
        \end{pmatrix}^T
    \end{equation*}
    for some $\beta_1,\ldots,\beta_{k-1}\in\K$. Therefore $\mathcal{R}(A^\flat,b^\flat)=n$.
\end{proof}

\begin{lemma}
\label{lem:simCont}
    Let $\K$ be a field and let $A_1,A_2\in\K^{n\times n}$ and $b_1,b_2\in\K^n$, such that the pairs $(A_1,b_1),(A_2,b_2)$ are controllable. If the characteristic polynomials of $A_1$ and $A_2$ are the same, then the pairs $(A_1,b_1),(A_2,b_2)$ are similar.
\end{lemma}

\begin{proof}
    Let us have a pair
    $$A^\dagger=(A^\flat)^T=
    \begin{pmatrix}
        0 & 0 & \cdots & 0 & \alpha_1 \\
        1 & 0 & \cdots & 0 & \alpha_2 \\
        0 & 1 & \cdots & 0 & \alpha_3 \\
        \vdots & \vdots & \ddots & \vdots & \vdots \\
        0 & 0 & \cdots & 1 & \alpha_n
    \end{pmatrix}\qquad
    b^\dagger=
    \begin{pmatrix}
        1 \\ 0 \\ 0 \\ \vdots \\ 0
    \end{pmatrix}\ .$$
    The characteristic polynomial of the matrix $A^\dagger$ is the same as the one of the matrix $A^\flat$ since transposing a matrix preserves its characteristic polynomial. Therefore, by Cayley-Hamilton theorem and by Lemma \ref{lem:flatCharPol}, it holds that
    $$O=\chi_{A^\dagger}(A)=\chi_{A^\flat}(A)=A^n-\alpha_nA^{n-1}-\ldots-\alpha_2A-\alpha_1I_n\ ,$$
    implying
    $$A^n=\alpha_nA^{n-1}+\ldots+\alpha_2A+\alpha_1I_n\ .$$
    It then follows 
    $$\mathbf{R}(A,b)A^\dagger=
        \begin{pmatrix} 
            b & Ab & \ldots & A^{n-1}b
        \end{pmatrix}
    A^\dagger=
        \begin{pmatrix}
            Ab & A^2b & \ldots & A^nb
        \end{pmatrix}
    =A\mathbf{R}(A,b)\ .$$
    By the controllability of the pair $(A,b)$, the column space of the matrix $\mathbf{R}(A,b)$ is of dimension $n$, which means, that the matrix is invertible. Therefore, we can write 
    $$A=\mathbf{R}(A,b)A^\dagger\mathbf{R}(A,b)^{-1}\ .$$
    We see that the matrices $A$ and $A^\dagger$ are similar. It is also true that $$\mathbf{R}(A,b)b^\dagger=b\ .$$ Therefore $(A,b)\sim(A^\dagger,b^\dagger)$.

    Since the pair $(A^\dagger,b^\dagger)$ depends only on the characteristic polynomial of the matrix $A$, we conclude by transitivity of the matrix similarity, that any two controllable pairs with the same characteristic polynomials are similar to each other.
\end{proof}

\begin{cor}
\label{cor:controllerForm}
    If the \termdef{single-input} ($m=1$) pair $(A,b)$ is controllable, then it is similar to its controller form.
\end{cor}

\begin{proof}
    Follows from Lemmas \ref{lem:flatCharPol}, \ref{lem:flatControllable} and \ref{lem:simCont}. 
\end{proof}

\begin{theorem}
    Let $\K$ be a field. Let $A\in\K^{n\times n}$, $B\in\K^{n\times m}$. The assignable polynomials for the pair $(A,B)$ are precisely of the form $$\chi_{AB+F}=\chi\chi_u$$ where $\chi$ is an arbitrary monic polynomial of degree $r=\text{dim}\mathcal{R}(A,B)$ and $\chi_u$ is the uncontrollable part of the assignable polynomial.

    In particular, the pair $(A,B)$ is controllable if and only if every nth degree monic polynomial can be assigned to it.
\end{theorem}

\begin{proof}
    By Theorem \ref{theorem:decomp} and Lemma \ref{lem:simPairsAssignablePolynomial} we can assume that the pair $(A,B)$ is in the same form as $(\widetilde{A},\widetilde{B})$ in (\ref{eq:decomp}). Let us write $F=(F_1,F_2)\in\K^{m\times n}$, where $F_1\in\K^{m\times r}, F_2\in\K^{m\times (n-r)}$. Then 
    \begin{align*}
        A+BF&=
        \begin{pmatrix}
            A_1 & A_2 \\
            0   & A_3
        \end{pmatrix}
        +
        \begin{pmatrix}
            B_1 \\
            0
        \end{pmatrix}
        \begin{pmatrix}
            F_1 & F_2
        \end{pmatrix}
        =
        \begin{pmatrix}
            A_1 & A_2 \\
            0   & A_3
        \end{pmatrix}
        +
        \begin{pmatrix}
            B_1F_1 & B_1F_2 \\
            0 & 0
        \end{pmatrix}
        \\
        &=
        \begin{pmatrix}
            A_1+B_1F_1 & A_2+B_1F_2 \\
            0 & A_3
        \end{pmatrix}
    \end{align*}
    It follows 
    $$\chi_{A+BF}=\chi_{A_1+B_1F_1}\chi_{A_3}=\chi_{A_1+B_1F_1}\chi_u$$
    We see that any assignable polynomial is a multiple of the uncontrollable part $\chi_u$.

    Conversely, we want to show that the first factor can be made arbitrary by a suitable choice of $F_1$. This makes sense only for $r>0$, otherwise the assignable polynomial is equal to $\chi_u$, which cannot be changed by modifying the matrix $F$. Assume that we are given a monic polynomial $\chi$. If we find such a matrix $F_1$ that 
    $$\chi_{A_1+B_1F_1}=\chi\ ,$$
    then by putting $F=(F_1,0)$ we get the desired characteristic polynomial, that is, $\chi_{A+BF}=\chi\chi_u$. Since the pair $(A_1,B_1)$ is controllable as shown in Lemma~\ref{lem:A_1B_1controllable}, it is sufficient only to prove that controllable systems can be assigned an arbitrary monic polynomial $\chi$ of respective degree. Hence, from this point on, we assume that the pair $(A,B)$ is controllable.

    We will first prove the theorem for $m=1$ and then we will generalize it. That will conclude the proof.

    Let $m=1$. By Lemma \ref{lem:simPairsAssignablePolynomial} and Corollary \ref{cor:controllerForm} we can consider the pair $(A,b)$ to be in the controller form. For a vector 
    \begin{equation*}
        f=\begin{pmatrix}
            f_1&f_2&\ldots&f_n
        \end{pmatrix}
    \end{equation*}
    we have 
    \begin{align*}
        A+bf&=
        \begin{pmatrix}
			0 & 1 & 0 & \ldots & 0 \\
			0 & 0 & 1 & \ldots & 0 \\
			\vdots & \vdots & \vdots & \ddots & \vdots \\
			0 & 0 & 0 & \ldots & 1 \\
			\alpha_1 & \alpha_2 & \alpha_3 & \ldots & \alpha_n
        \end{pmatrix}
        +
        \begin{pmatrix}
            0 \\
            0 \\
            0 \\
            \vdots \\
            1
        \end{pmatrix}
        \begin{pmatrix}
            f_1&f_2&\ldots&f_n
        \end{pmatrix}
        \\
        \\
        &=
        \begin{pmatrix}
			0 & 1 & 0 & \ldots & 0 \\
			0 & 0 & 1 & \ldots & 0 \\
			\vdots & \vdots & \vdots & \ddots & \vdots \\
			0 & 0 & 0 & \ldots & 1 \\
            \alpha_1+f_1 & \alpha_2+f_2 & \alpha_3+f_3 & \ldots & \alpha_n+f_n
        \end{pmatrix}\ .
    \end{align*}
    Following from Lemma \ref{lem:flatCharPol}, one can see that given a monic polynomial
    $$\chi=s^n-\beta_ns^{n-1}-\ldots-\beta_2s-\beta_1\ ,$$
    we can choose
    $$f=\begin{pmatrix}
        \beta_1-\alpha_1&\beta_2-\alpha_2&\ldots&\beta_n-\alpha_n
    \end{pmatrix}\ ,$$
    and the equality $\chi_{A+bf}=\chi$ will be satisfied. We have shown that for the case where $m=1$, the controllable pair $(A,b)$ can be assigned an arbitrary monic polynomial of degree $n$.

    For the case where $m>1$, we choose any vector $v\in\K^m$ satisfying that $Bv\neq \nullvector$ and put $b=Bv$. For any $f\in\K^{1\times n}$ and for any matrix $G\in\K^{m\times n}$, it then holds
    $${A+BG+bf}={A+BG+Bvf}={A+B(G+vf)}$$
    and therefore, if we put $F=G+vf$, we obtain 
    $$\chi_{A+BG+bf}=\chi_{A+BF}\ .$$
    This implies that any polynomial that can be assigned to the pair $(A+BG,b)$ can also be assigned to the pair $(A,B)$. Since we have proved the theorem for a controllable pair where $m=1$, the proof will be concluded by showing that there exists such a matrix $G$ that the pair $(A+BG,b)$ is controllable.
       
    Let us have a sequence of linearly independent vectors $Bv=x_1,\ldots,x_k$, of length $n$, where
    \begin{equation}
    \label{eq:indSeq}
        x_{i}=Ax_{i-1}+Bu_{i-1},\ i\in\{2,\ldots,k\}
    \end{equation}
    for some $u_i\in\K^m$, and assume that $k$ is as large as possible. We denote the span of $\{x_1,\ldots,x_k\}$ by $\mathcal{V}$. By the maximality of $k$ we have $x_{k+1}\in\mathcal{V}$, which implies that
    \begin{equation}
    \label{eq:inV}    
        Ax_k+Bu=x_{k+1}\in\mathcal{V}
    \end{equation}
    for any $u\in\K^m$. Therefore, in particular for $u=\nullvector$, we get 
    \begin{equation}
    \label{eq:AxkinV}
        Ax_k\in\mathcal{V}\ .
    \end{equation}
    It follows by (\ref{eq:inV}) and (\ref{eq:AxkinV}), that for any $u\in\K^m$ it holds
    $$Bu=x_{k+1}-Ax_k\in\mathcal{V}\ ,$$
    which implies that the column space $\mathcal{B}=\text{Im}B$ is included in $\mathcal{V}$. Following from this and the equality (\ref{eq:indSeq}), we have 
    $$Ax_{i-1}=x_i-Bu_{i-1}\in\mathcal{V}$$
    for $i\in\{2,\ldots,k\}$. This result together with the equation (\ref{eq:AxkinV}) shows that for any $i\in\{1,\ldots,k\}$ it is true that $Ax_i\in\mathcal{V}$. This means, that $\mathcal{V}$ is an $A$-invariant subspace containing $\mathcal{B}$. Using these two facts one can see that 
    \begin{align*}
        \mathcal{B}&\subseteq\mathcal{V} \\
        A\mathcal{B}&\subseteq A\mathcal{V}\subseteq\mathcal{V} \\
        A^2\mathcal{B}&\subseteq A(A\mathcal{V})\subseteq\mathcal{V} \\
        &\vdotswithin{\subseteq} \\
        A^{n-1}\mathcal{B}&\subseteq\mathcal{V}
        \ .
    \end{align*}
    Therefore, it holds
    $$\mathcal{R}(A,B)=\text{Im}(B|AB|A^2B|\ldots|A^{n-1}B)\subseteq\mathcal{V}\ .$$
    By the controllability of the pair $(A,B)$, we obtain
    $$n=\text{dim}\mathcal{R}(A,B)\leq\text{dim}\mathcal{V}=k\leq\text{dim}\K^n=n\ .$$
    This implies that $k=n$, $\mathcal{V}=\K^n$.

    Let us define a linear mapping $g\colon\mathcal{V}\to\mathcal{B}\subseteq \mathcal{V}$ by the equation $g(x_i)=u_i$ for every $i\in\{1,\dots,n-1\}$, where $u_i$ is  such an element that $Ax_i+Bu_i=x_{i+1}$, and we define $g(x_n)$ arbitrarily. This definition is correct and unique since the vectors $x_i$ form a basis of $\mathcal{V}$ \citep[see][Tvrzení 6.4]{Barto}. Let $G$ be the matrix of the linear mapping $g$ with respect to the standard basis. Then for every $i\in\{1,\ldots,n-1\}$ we have
    $$(A+BG)x_i=Ax_i+BGx_i=Ax_i+Bu_i=x_{i+1}\ .$$
    It follows 
    $$\mathbf{R}(A+BG,x_1)=(x_1|(A+BG)x_1|\cdots|(A+BG)^{n-1}x_1)=(x_1|x_2|\cdots|x_n)\ .$$
    Finally, by the linear independence of the the vectors $x_1,\ldots,x_n$, it holds that $\text{dim}\mathcal{R}(A+BG,x_1)=n$. We have shown that the pair $\mathcal{R}(A+BG,Bv)$ is controllable, and thus the proof is concluded.
\end{proof}

\chapter*{Conclusion}
\addcontentsline{toc}{chapter}{Conclusion}

In this thesis, we described and proved notions and relations which were needed to fully understand the pole shifting theorem. We started with the definition of continuous linear autonomous systems. We showed that the system is stable if the eigenvalues of its coefficient matrix have negative real parts.
Next, we derived the rank condition for controllability using discrete-time systems and the Cayley-Hamilton theorem. After that, we proved that this condition also holds for continuous-time systems. Then, we proved that the characteristic polynomial of the coefficient matrix of the system splits uniquely into its controllable and uncontrollable parts. 

Subsequently, we formulated and proved the pole shifting theorem. The theorem states that given a monic polynomial of degree equal to the rank of the reachability matrix, we can always find such a feedback matrix that the characteristic polynomial of the resulting coefficient matrix is equal to the given polynomial times the uncontrollable part.

One of the corollaries is that if we work with controllable system, we can, using the pole shifting theorem, find such a feedback matrix that the system is stable. 

%%% Bibliography
\include{bibliography}

%%% Abbreviations used in the thesis, if any, including their explanation
%%% In mathematical theses, it could be better to move the list of abbreviations to the beginning of the thesis.
%\chapwithtoc{List of Abbreviations}

\openright
\end{document}